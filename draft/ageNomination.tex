\documentclass[a4paper, 11pt]{article}

% packages
\usepackage[utf8]{inputenc}
\usepackage{amsmath}
\usepackage{amsfonts}
\usepackage{array}
\usepackage{booktabs}
\usepackage{chngcntr}
\usepackage{colortbl}
\usepackage{dcolumn}
\usepackage{floatrow}
\usepackage{geometry}
\usepackage{graphicx}
\usepackage{hyperref}
\usepackage{multirow}
\usepackage{natbib}
\usepackage[normalem]{ulem}
\usepackage{ntheorem}
\usepackage{pdflscape}
\usepackage{rotating}
\usepackage{setspace}
\usepackage{subfigure}
\usepackage{tabularx}
\usepackage{threeparttable}
\usepackage[numbib, nottoc, notlot, notlof]{tocbibind}
\usepackage{wrapfig}
\usepackage{xcolor}

% page styling
\geometry{
  a4paper,
  left=1in,
  right=1in,
  bottom=1.2in,
  top=1.2in
}
\pagestyle{plain}
\pagenumbering{arabic}
\linespread{1.5}
\hypersetup{
  colorlinks,
  linkcolor={red!50!black},
  citecolor={blue!50!black},
  urlcolor={blue!80!black}
}
\theoremseparator{:}
\newtheorem{hyp}{Hypothesis}
\counterwithin*{hyp}{section}

\renewcommand{\thefootnote}{\fnsymbol{footnote}}

\title{
	Party Nomination Strategy and its Representational Consequences in Interactive Mixed-Member Majoritarian Systems
	\footnotemark{}
	\footnotetext[1]{This paper was previously entitled and circulated as ``Youth Underrepresentation and Parties' Nomination Strategy in Mixed-Member Electoral Systems." Earlier versions of this paper were presented at the 2024 summer meeting of the Japanese Society for Quantitative Political Science (JSQPS) and the 2024 Annual Meeting of Americal Political Science Association (APSA). I thank Dan Smith for sharing the latest version of his data, and Serika Atsumi, Yuki Atsusaka, Amy Catalinac, Kentaro Fukumoto, Yusaku Horiuchi, Rieko Kage, Junko Kato, Kenneth McElwain, Kento Ohara, Mayuko Toba, Masahiro Yamada, Hironao Yoda for their comments.}
}

\author{
	Dai Sasaki
	\thanks{Master's Student, Graduate Schools for Law and Politics, The University of Tokyo. Email: daichansama12@g.ecc.u-tokyo.ac.jp; Website: https://dxxsxsxkx.github.io.}
}

\date{
	First Version: 28 Jun, 2024 \\
	This Version: 7 Dec, 2024 
}

\begin{document}

\maketitle

\renewcommand{\thefootnote}{\arabic{footnote}}
\setcounter{footnote}{0}

\begin{abstract} 
I argue that interactive mixed-member majoritarian systems (interactive MMMs), a variant of mixed member systems where parties can nominate the same candidates in both majoritarian and PR tiers (dual listing), diminish the representational advantages commonly associated with PR systems. Analyzing comprehensive, candidate-level data of Japan's lower house elections, I show that parties give higher list ranks to senior candidates, incumbents, and dual-listed candidates under the interactive MMM. Furthermore, incumbents are more likely to be dual-listed than non-incumbents. These patterns apply across parties, but are less applicable to situations of intra-party disputes and government transitions, where seniors and incumbents may give their way to newcomers. My analysis suggests that interactive MMMs sustain representational inequalities between groups by reducing the electoral prospects of newcomers and making legislative turnover less frequent.
\end{abstract}

\newpage

\section{Introduction}

Scholars of electoral systems have long recognized that proportional representation (PR) systems help achieve higher levels of minority representation than majoritarian systems \citep{matlandContagionWomenCandidates1996, matlandWomensRepresentationNational1998, meserveGenderIncumbencyParty2020}. The representational advantage of PR over majoritarian systems originate in the voters' evaluation of candidates and parties' response to it. Under (closed) PR, voters generally vote for a list of candidates, not a specific candidate, which means personal votes are less frequent than in majoritarian systems. In such circumstances, parties have incentives to appeal to a broader set of voters by diversifying their lists. PR also falicitates the representation of interests that are geographically scattered \citep{ogradyHowGeographicClustering2024, teelePoliticalGeographyGender2024} or of relatively lower salience \citep{wiedemannRedistributivePoliticsSpatial2024}. As its name suggests, PR improves the proportionality between votes and seats and promotes the representation of benefits that are oft-underrepresented in majoritarian systems. 

Then, how is this representational advantage altered when PR is embedded within an electoral system? This paper focuses on the party nomination strategy in mixed-member majoritarian (MMM) systems, a type of mixed-member (MM) electoral systems \citep{shugartMixedmemberElectoralSystems2003} that combines majoritarian and proportional systems.\footnotemark{} In the other pattern of MM, mixed-member proportional  (MMP) systems, parties' seat shares depend solely on the PR tier and the majoritarian tier decides who gets elected within each party. In contrast, each party's tally in an MMM is the sum of the seats it obtains in the majoritarian and PR tiers. I analyze how parties' nomination strategy in MMMs affects the PR's representational advantage.

\footnotetext{Of the 36 countries that employ MM systems, 27 utilize the MMM variant \citep{catalinacGeographicallyTargetedSpending2021}. These countries span diverse regions and include Andorra, Guinea, Japan, Italy, Libya, Lithuania, Mauritania, Monaco, Nepal, Niger, Pakistan, Panama, the Philippines, Russia, Senegal, Seychelles, Sudan, Taiwan, Tajikistan, Ukraine, Venezuela, Zimbabwe, Hungary, Mexico, Cameroon, and Chad. Some countries recently departed from MMM: for example, Georgia transitioned to a pure-PR following its 2017 constitutional reform, with this change taking effect starting in the 2024 parliamentary election \citep{internationalfoundationforelectoralsystemsElectionsGeorgia20242024}. South Korea adopted a mixed-member proportional (MMP) system for its 2024 National Assembly elections \citep{internationalidea2024SouthKorean2024}.}

This paper shows that institutional arrangements in MMMs could diminish the representational advantage of the PR tier. Specifically, I argue that parties tend to prioritize their senior members and incumbents over others in a variant of MMMs, which I call ``interactive mixed-member systems (interactive MMM)."\footnotemark{} Interactive MMMs allow parties to nominate the same candidates in both majoritarian and PR tiers (dual listing). Even in the PR tier of interactive MMMs, parties still possess incentives to diversify their party lists, because voters vote for a list, not a candidate. However, the following mechanisms would prevail. 

\footnotetext{This terminology helps distinguish the electoral system this paper focuses on from ordinal MMMs or ``parallel systems," where the majoritarian and PR tiers function separately.}

First, the post-election goals of parties incentivize them to maximize the election prospects of senior candidates. Parties are not solely focused on maximizing vote shares but also on electing candidates who can effectively advance their policy goals in parliament, secure key positions within the party, or, for majority-seeking parties, take on government roles. Senior politicians typically possess more political resources than junior politicians, allowing them to perform more effectively in legislative activities and policy-making. In interactive MMMs, parties can provide “insurance tickets” to candidates through dual listing, whereby those losing in the majoritarian tiers may get elected via the PR tier. This mechanism offers senior politicians or incumbents a second chance to be elected.

Another reason parties extensively utilize this mechanism is that dual listing enables parties to monitor candidates' mobilization efforts. Candidates who run solely in the PR tier have less need to cultivate a personal votes and may reduce their campaign efforts by free-riding on others’ work. In contrast, candidates running in both tiers must build personal support in their districts, reducing concerns for the party about their commitment. Furthermore, this mechanism helps parties avoid allegations of favoritism towards specific candidates within their ranks.

This paper analyzes party nomination strategies under interactive MMMs using the case of Japan’s lower house (House of Representatives) elections. Since 1994, Japan’s lower house elections have used an interactive MMM system that permits dual listing. Drawing on data from all candidates who ran in the PR tier from 1994 to 2021, the study examines how parties make their nominations. I conduct the analysis at three levels: aggregate, party, and party-election levels.

The findings reveal that parties tend to favor senior and incumbent candidates when constructing their PR lists, often placing these candidates in higher-ranking positions to improve their chances of being elected. Furthermore, incumbents are shown to have a higher likelihood than non-incumbents of being dual-listed. While not every party demonstrates identical behaviors, each party exhibits some form of preference for senior and incumbent candidates, reflecting the strong incentives created by the system. However, the study also finds that this trend can weaken during periods of internal party conflict or when a governing party transitions to the opposition, leading to a reshuffling of candidates. 

This paper contributes to the relevant literatures in two ways. First, it builds on the extensive body of research examining parties' candidate nomination strategies in the context of electoral systems, particularly PR systems \citep{andrePartyNominationStrategies2017, buisseretPartyNominationStrategies2022, crutzenModelTeamContest2020, dancygierElectoralRulesElectoral2014, hoboltSelectionSanctioningEuropean2011, nemotoLocalismCoordinationThree2013}. Prior studies have demonstrated that under closed-list or flexible-list PR, the ranking of candidates on party lists reflects the party’s prioritization of those candidates. These studies consistently suggest that parties place “stronger” candidates—defined by factors such as cognitive ability, education, or the ability to attract preference votes—at the top of their lists \citep{buisseretPartyNominationStrategies2022, coxMoralHazardElectoral2021}. However, the question of how candidate nomination strategies might change when the proportional formula is embedded within a mixed-member electoral system, rather than used independently, has received little attention. Although this paper does not address all aspects of party nomination strategies under MMM broadly, it provides a meaningful contribution by identifying specific patterns of candidate prioritization under particular institutional arrangements. 

Second, and most importantly, this paper demonstrates that the representative advantages typically associated with PR systems can be undermined under interactive MMMs. In this system, parties are incentivized to provide insurance to senior and incumbent candidates through dual listing. This practice, while strategic for ensuring the election of experienced candidates, inadvertently restricts the entry of new candidates. Consequently, the nomination strategies of parties under interactive MMMs may delay or prevent the inclusion of underrepresented groups who are already disadvantaged at the time the system is introduced. To illustrate this, the paper uses the example of youth underrepresentation in Japan. While PR systems are often praised for their ability to promote the representation of minorities, cross-national comparisons have shown that PR systems generally result in a higher proportion of younger legislators compared to majoritarian systems \citep{stockemerAgeRepresentationParliaments2018, stockemerAgeInequalitiesPolitical2023}. However, in mixed-member systems, if the institutional design is not carefully considered, the inherent representational benefits of PR can be diminished or entirely lost.

This paper proceeds as follows. Section \ref{sec: the} discusses the theoretical backgrounds of the research, presents the main arguments, and formulate hypotheses. It discusses how parties would render their nomination strategies where the two tiers are institutionally interacted. I also outline Japan's mixed-member system with a focus on dual listing. Section \ref{sec: emp} presents the data and empirical strategies. Section {sec: res} shows the results of the analysis. It presents aggregate-level and party-specific evidence, as well as election- and party-specific analysis. Section \ref{sec: dis} discusses the implication of the results, focusing on legislative turnover and representation. 

\section{Theory} \label{sec: the}

\subsection{Theoretical Expectations}

It remains unclear how party nomination strategies and patterns uncover in mixed-member systems, in contrast to ``pure" electoral systems. MM systems have often been regarded to deliver ``the best of both worlds," that is, majoritarian and PR \citep{shugartMixedmemberElectoralSystems2003, hiranoPolicyPositionsMixed2011}. This perception stems from the theoretical argument that MM helps avoid the extreme outcomes associated with pure systems on both inter-party and intra-party dimensions. On the inter-party dimension, MM combines the features of majoritarian systems, which favor two-party dominance, and PR systems, which facilitate the representation of smaller parties. As a result, MM is thought to mitigate the vote-seat disproportionality inherent in majoritarian systems. For example, under a mixed system, it is expected to be less likely for a party to secure a legislative majority without achieving an absolute majority of the vote, as frequently observed under pure majoritarian systems. On the intra-party dimension, mixed systems are believed to balance competing incentives by preserving and fostering ties between voters and candidates through the nominal tier while simultaneously enhancing party cohesion through the list tier.

However, scholars have noted that mixed systems do not necessarily occupy a middle ground between the two pure systems in all respects. In fact, the interaction between the two tiers within mixed systems can create what is referred to as a “contamination effect,” where one tier influences the dynamics of the other. Much of the existing research on mixed systems has focused on the implications of these contamination effects for party systems \citep{bawnComparativeTheoryElectoral2003, coxInteractionEffectsMixedMember2002, ferraraMixedElectoralSystems2005, herronContaminationEffectsNumber2001, nishikawaMixedElectoralRules2004, moserMixedElectoralSystems2004}. Key areas of inquiry include how the actual number of parties or candidates deviates from the theoretical predictions of Duverger’s law, the nature of the relationship between vote shares and seat shares, and the responsiveness of candidates to different types of electoral incentives.

In contrast to previous studies, however, this article is concerned with how MMM systems are positioned relative to pure systems in terms of candidate nomination strategies and representativeness. Specifically, it examines whether the institutional features of MMM systems lead to patterns that align more closely with majoritarian, pure PR, or something distinct. To the author’s knowledge, few studies have directly addressed these questions with regard to MMMs. While related research exists, it often focuses on other mixed systems, such as MMP systems, or on broader aspects of electoral outcomes rather than nomination strategies or representativeness. For instance, \citet{fortin-rittbergerGenderEqualBundestagImpact2013} found that women’s descriptive representation improved under Germany’s MMP system. However, their findings highlighted the role of voluntary quotas adopted by political parties, rather than the electoral system itself, in improving gender representation.

This paper argues that the institutional configuration of MMMs creates incentives for parties to favor senior candidates in the PR tier. Specifically, these incentives arise under interactive MMM systems, which allow dual listing—nominating the same candidate in both the majoritarian and proportional tiers. While dual listing is commonly associated with MMPs, such as in Germany and New Zealand \citep{konoDualCandidacyRemedy2020, electoralcommisionofnewzealandNewZealandsElectoral2014}, it is also utilized in MMM systems as in Japan, Hungary, Italy, and Lithuania \citep{pappDualCandidacySource2021, lemondeHowDoesItalian2022, thecentralelectoralcommissionoftherepublicoflithuaniaElectionsReferendum2024}. This paper contends that the use of dual listing in interactive MMM plays a key role in shaping party nomination strategies, particularly by encouraging the prioritization of experienced candidates.

Existing research provides insights into how dual listing operates in mixed systems, though much of this work has focused on its effects rather than its implications for candidate selection. For example, \citet{mckeanJapansNewElectoral2000} demonstrate that dual listing can reduce the ratio of candidates to available seats, a pattern that reflects the strategic consolidation of party resources. Meanwhile, \citet{kraussReverseContaminationBurning2012} highlight that parties and candidates can use PR list rankings to send signals to voters, encouraging strategic voting in the majoritarian tier for those with lower list ranks in the PR tier. These studies underscore the multifaceted role that dual listing plays in shaping electoral dynamics, but they have not fully examined how this mechanism influences the representational quality of proportional tiers within MMM systems. By focusing on this gap, the present study aims to shed light on the broader implications of interactive MMM for candidate nomination strategies and the diversity of legislative representation.

How does dual listing influence party nomination strategies? This paper argues that the allowance of dual listing in interactive MMMs incentivizes parties to provide “insurance tickets” to senior politicians. For political parties, it is crucial not only to maximize the number of seats they win but also to determine which candidates fill those seats. This dual priority stems from the nature of political parties, which function not just as vote-gathering mechanisms during elections but also as organizations with post-election legislative and governance goals \citep{matakosElectoralInstitutionsIntraparty2024}. In this context, parties face two competing incentives: on one hand, they aim to continue electing senior candidates who can occupy leadership roles within the party or government; on the other hand, they must foster the election of new candidates to groom future senior members. Given that parties estimate the number of seats they can realistically win, their nomination strategy can be seen as solving an optimization problem under the constraint of attainable seats.

As policy-driven entities, parties rank candidates based on seniority, as senior members typically possess critical resources for policymaking. These resources include expertise in drafting policies, legislative process knowledge, negotiation skills with other parties, coordination with bureaucracies, and connections with stakeholders. Generally, politicians acquire these resources through experience, creating a positive correlation between tenure and resource accumulation. A similar dynamic exists between incumbent and non-incumbent candidates. When possible, parties reflect these priorities in their nomination strategies \citep[p.1085]{reedNominationProcessJapans1995}. Dual listing provides an avenue for parties to act on these incentives. By nominating senior politicians or incumbents on both tiers, parties can offer them an “insurance ticket.”

However, if parties merely valued senior candidates, they could simply place them at the top of PR lists without using dual listing. In PR nominations, parties prioritize candidates who can attract substantial party votes. While nominating nationally recognized celebrities is one approach, their competency may be uncertain. A more reliable option is to nominate veteran politicians with proven abilities and high name recognition \citep{reedNominationProcessJapans1995}. Nevertheless, the use of dual listing is driven by the institutional features of mixed-member systems.

In MMMs, winning in single-member districts (SMDs) is critical for majority-seeking parties \citep{bawnComparativeTheoryElectoral2003, catalinacGeographicallyTargetedSpending2021}, but candidates capable of succeeding in SMDs are scarce. In interactive MMM, parties use dual listing to provide a “second chance” to promising SMD candidates, thereby managing long-term priorities. This approach reassures candidates by offering a backup plan, ensuring that even if they lose in the SMD, they can still win via the PR tier and maintain their ties with voters as incumbents. At the same time, parties ensure these candidates remain motivated by requiring their SMD candidacy \citep{bawnComparativeTheoryElectoral2003}. This method avoids accusations of favoritism while still strategically managing resources. For example, parties can nominate candidates for safe PR seats while deploying them as challengers in competitive SMDs \citep{reedNominationProcessJapans1995}. However, this strategy risks perceptions of nepotism among other party members and does not guarantee that the candidate will excel in their legislative role.

For certain candidates, such as those with unassailable reputations, established SMD performance, or those nearing retirement and preparing successors, dual listing’s “carrot and stick” approach may be unnecessary. These candidates can be directly placed in PR seats or contested freely in SMDs without additional oversight. \footnotemark{}

\footnotetext{In Japan, former prime ministers Yasuhiro Nakasone and Kiichi Miyazawa were once granted the top position on the PR list \citep{theasahishimbunNakasoneMiyazawaRyou2006}.}

When dual listing is allowed, parties face two key options for PR nominations: providing insurance to existing SMD candidates or nominating entirely new candidates. These options are, to some extent, mutually exclusive. Party resources, such as campaign funding and organizational support, are finite, and the number of candidates parties can field simultaneously is limited. Parties also align their nominations with the expected number of seats they believe they can win, making excessive over-nomination rare. Consequently, when parties use dual listing to enhance the electoral prospects of existing members, they tend to nominate fewer new candidates. Even when new candidates are nominated, their chances of election are not necessarily high. In closed-list PR systems, a candidate’s position on the list directly correlates with their likelihood of election: the higher the rank, the greater the chances. As long as parties prioritize senior or incumbent candidates, new candidates are likely to be placed in lower, less winnable positions on the list.

Based on the above discussion, five hypotheses are proposed: 

\begin{hyp}[H\ref{hyp:first}] \label{hyp:first}
Senior candidates (i.e., those with more terms in office) are assigned higher positions on the proportional representation (PR) list compared to less senior candidates.
\end{hyp}

\begin{hyp}[H\ref{hyp:second}] \label{hyp:second}
Incumbents are assigned higher positions on the PR list compared to non-incumbents.
\end{hyp}

\begin{hyp}[H\ref{hyp:third}] \label{hyp:third}
The more senior a candidate is, the more likely they are to be dual-listed.
\end{hyp}

\begin{hyp}[H\ref{hyp:fourth}] \label{hyp:fourth}
Incumbents are more likely to be dual-listed than non-incumbents.
\end{hyp}

\begin{hyp}[H\ref{hyp:fifth}] \label{hyp:fifth}
Dual-listed candidates are assigned higher positions on the PR list compared to non-dual-listed candidates.
\end{hyp}

These hypotheses are expected to hold across all political parties. For large parties, which are likely to secure numerous seats in each election and have a strong incentive to win government, there is a significant motivation to elect candidates capable of serving in cabinet positions. Large parties often have a wealth of experienced members and internal practices that allocate roles based on seniority, reinforcing their incentive to prioritize senior and incumbent candidates for dual listing and higher PR positions.

For smaller parties, while they may have an incentive to use PR lists to bring in younger politicians, they are also likely to utilize dual listing to ensure the election of their core members. Given the uncertainty of winning in single-member districts (SMDs), smaller parties often rely on their PR vote share to secure a stable number of seats. In MMMs, smaller parties may also exchange votes across tiers with larger parties \citep{catalinacGeographicallyTargetedSpending2021}. For example, smaller parties may support larger party candidates in SMDs in exchange for proportional votes. Under these circumstances, smaller parties have an incentive to provide insurance to senior and incumbent candidates by prioritizing them on PR lists or dual-listing them.

However, there are conditions under which Hypotheses 1 and 3 may be weaker: First, the incentive to insure senior candidates diminishes when a party loses its majority and falls out of government, as the need to fill key government roles decreases. Second, if a party enters an election amid factional disputes, the dominant faction may have an incentive to sideline members of opposing factions rather than granting them insurance through high PR positions or dual listing. 

\subsection{Case: Japan's Mixed-Member Majoritarian System}

This paper examines the above hypotheses using data from Japan’s House of Representatives elections under its MMM system. In 1994, Japan underwent electoral reform, adopting a MMM system that combines SNTV-SMD and PR tiers. When a new electoral system is introduced, political actors gradually adapt to its rules and incentives. Therefore, to accurately evaluate the behavioral patterns and incentive structures under a given system, it is crucial to analyze it after sufficient time has elapsed for adaptation. In this respect, Japan’s MMM system, implemented nearly 30 years ago, provides a suitable case for examining the long-term incentives created by this institutional framework. Furthermore, dual listing is widely utilized in Japan, garnering significant public and media attention. Figure \ref{fig:dual} illustrates the proportion of candidates elected via the PR tier who were dual-listed in each House of Representatives election. Since 2000, more than half of all PR-elected candidates have consistently been dual-listed. This trend highlights the strategic importance of dual listing under Japan’s MMM system, providing a rich context for analyzing how it shapes candidate selection and ranking strategies.

\begin{figure}[!htbp]
	\includegraphics[width = 0.9\textwidth]{../figure/paper/dual_nomination.pdf}
	\caption{Share of Dual-Listed Candidates among PR Winners}
	\label{fig:dual}
\end{figure}

MPs who secure a seat through dual listing (i.e., losing in the SMD but winning in the PR) are often derisively termed "zombie MPs." The prevalence of this "resurrection" pattern is frequently highlighted in media coverage, attesting to its common occurrence. Moreover, comprehensive candidate-level data for Japan \citep{reedReedSmithJapaneseHouse2017} enables detailed analysis of party candidate nomination patterns. To the best of our knowledge, no other country using an MMM system possesses such granular candidate data. These factors make Japan an exceptionally suitable case for analyzing candidate nomination patterns under an interactive MMM.

Prior to the electoral reform, Japan employed an SNTV-MMD system. A key feature of this system was the possibility of multiple candidates from the same party winning in a single MMD. This necessitated candidates to differentiate themselves from their party colleagues to secure personal votes, often leading to political corruption. The prevalence of such corruption was a catalyst for electoral reform.

The primary objective of electoral reform was to shift elections from a candidate-centered to a policy- and party-centered focus. The Eighth Electoral System Council, established in June 1989, submitted a reform proposal in April 1990. This proposal emphasized increasing the likelihood of government change and advocated for the adoption of a mixed-member majoritarian (MMM) system with a strong SMD component \citep{theasahishimbunShuuinNiShousenkyoku1990, yoshidaChusenkyokuseiKaraShosenkyokuhireidaihyou2018}. While the council also considered MMP \citep{theasahishimbunSenkyoseidoshinNoGijiroku1991}, and some parties (especially Komeito and Socialist Party) proposed similar systems, these were not adopted. Although the Kaifu Cabinet, which received the council's proposal, failed to implement the draft, the subsequent Hosokawa Cabinet enacted a nearly identical reform.

The new system combined SMD (SNTV-SSD) and PR tiers. Under this system, voters cast two ballots: one for an SMD candidate and another for a party list in the PR block. Personal votes could not be cast in the PR block (closed-list). The number of seats a party won was the sum of its SMD and PR seats. SMD seats were determined using SNTV, while PR seats were allocated to parties within each block using the d'Hondt method, and then distributed among candidates within each party.

Japan's MMM is an interactive system that permits dual listing \citep{pekkanenElectoralIncentivesMixedMember2006, reedJapaneseElectoralSystems2020}. Parties can nominate candidates for both the SMD and PR. Additionally, parties can assign the same list position to multiple dual candidates. Table \ref{tab:listStructure} presents possible list composition patterns in Japan's MMM. Parties can choose not to use dual listing and simply rank candidates from top to bottom (List A). Alternatively, they can dual list multiple candidates and assign the same rank to some (List B) or all (List C). As List B shows, it is not necessary to place all dual candidates at the top. It is also possible to create multiple strata among dual candidates (List D). However, it is not permissible to assign the same rank to candidates who are not dual listed (List E).

% created manually
\begin{table}[!htbp]
\centering
\begin{threeparttable}
\begin{tabular}{cccccccccc}
\toprule
\multicolumn{8}{c}{Valid} & \multicolumn{2}{c}{Invalid} \\
\cmidrule(lr){1-8} \cmidrule(lr){9-10}
\multicolumn{2}{c}{List A} & \multicolumn{2}{c}{List B} & \multicolumn{2}{c}{List C} & \multicolumn{2}{c}{List D} & \multicolumn{2}{c}{List E} \\
\cmidrule(lr){1-2} \cmidrule(lr){3-4} \cmidrule(lr){5-6} \cmidrule(lr){7-8} \cmidrule(lr){9-10}
Rank & Dual & Rank & Dual & Rank & Dual & Rank & Dual & Rank & Dual \\
\cmidrule(lr){1-2} \cmidrule(lr){3-4} \cmidrule(lr){5-6} \cmidrule(lr){7-8} \cmidrule(lr){9-10}
1 & - & 1 & - & 1 & \checkmark & 1 & \checkmark & 1 & - \\
2 & - & 2 & \checkmark & 1 & \checkmark & 1 & \checkmark & 1 & - \\
3 & - & 2 & \checkmark & 1 & \checkmark & 3 & \checkmark & 3 & - \\
4 & - & 2 & \checkmark & 1 & \checkmark & 3 & \checkmark & 4 & \checkmark \\
5 & - & 2 & \checkmark & 5 & - & 5 & - & 5 & \checkmark \\
6 & - & 6 & \checkmark & 6 & - & 6 & - & 6 & - \\
7 & - & 7 & \checkmark & 7 & - & 7 & - & 7 & - \\
8 & - & 8 & \checkmark & 8 & - & 8 & - & 8 & - \\
\bottomrule
\end{tabular}
\begin{tablenotes}[flushleft]
  \scriptsize{
    \item \textit{Note.} This table presents five hypothetical party lists that may be submitted in the PR tier of Japan's mixed member majoritarian system. Dual-listed candidates are denoted by \checkmark. Lists A, B, C, and D are all valid. List E is invalid, as pure-PR candidates cannot be ranked equal. 
  }
\end{tablenotes}
\end{threeparttable}
\caption{Valid and Invalid List Structures in Japan's Interactive MMM system}
\label{tab:listStructure}
\end{table}



















The final list order, which determines the allocation of seats within a party, is adjusted based on the SMD election results. First, SMD winners among dual candidates are removed from the list. Second, candidates with the same rank are sorted based on their performances in the SMD tier.\footnotemark{} The final seat allocation within the party is then based on this revised list order.

\footnotetext{Technically, candidates are reranked according to what is called ``close-loss rate." Let $i$ be a SMD-losing dual-listed candidate in the district $d$. Candidate $i$'s close loss rate, $\text{Loss}_{i}$, is calculated as 
  \[
    \text{Loss}_{i} = \frac{\text{Vote}_i}{\text{max}(\text{Vote}_d)}, 
  \]
\noindent where $\text{max}(\text{Vote}_d)$ denotes the largest number of votes obtained in the district $d$ (i.e., that of the winner). 
}

The exact process by which this system was included in the new electoral system is unclear. The Eighth Electoral System Council's proposal already included this provision \citep{yoshidaChusenkyokuseiKaraShosenkyokuhireidaihyou2018}. However, the council's minutes are not publicly available, making it impossible to determine who proposed the inclusion of this provision or at what stage \citep{theasahishimbunSenkyoseidoshinNoGijiroku1991}.

\section{Data and Method} \label{sec: emp}

To test the hypotheses, I use the Japanese House of Representatives Elections Dataset (JHRED; \citet{reedReedSmithJapaneseHouse2017}). Specifically, I analyze the nomination patterns of candidates running in the PR tier of the post-reform general election between 1996 and 2021. This criterion leaves 7,754 candidates across nine elections (2000, 2003, 2005, 2009, 2012, 2014, 2017, and 2021). Table \ref{tab:stats} in the appendix presents candidate-level summary statistics. 

I estimate a series of logistic and negative binomial models. For Hypotheses \ref{hyp:first}, \ref{hyp:second}, and \ref{hyp:fifth}, the dependent variable is the list rank of candidates. I estimate negative binomial models, as the rank of a candidate in the PR list can be treated as count data, representing how many candidates are ranked higher than the given candidate. For Hypotheses \ref{hyp:third} and \ref{hyp:fourth}, the dependent variable is the dual listing dummy that takes a value of one for dual-listed candidates. Logit models are estimated for this variable. I also use estimate logistic models where the dependent variable is the tie dummy, which equals unity for candidates that are dual-listed and are ranked the same as other candidates (c.f., Lists B, C, and D in Table \ref{tab:listStructure}). 

In addition to the above covariates, I also include a tie dummy, where a value of 1 is assigned to candidates who are dual-listed and are ranked the same as other candidates, to distinguish patterns among dual-listed candidates. This dummy variable is interacted with the dual listing dummy, the number of past wins, and incumbent dummy, where applicable. 

I include a number of covariates in the models, including a female dummy, district magnitude, and two-way fixed effects for election year and party. I control for candidates' gender because parties have incentives to give female candidates favorable positions in PR lists to appeal to a broader set of voters \citep{salmondProportionalRepresentationFemale2006, chiruValueLegislativeElectoral2017}. I also account for block magnitudes, as the length of a list depends on the maximum number of candidates elected. Magnitudes vary across time and region: see Table \ref{tab:distM} in the appendix for details. Party FE accounts for different nomination patterns among parties. For example, \citet{fujimuraShousenkyokuHireidaihyouHeiritsusei2012} finds different patterns in allocating the parties' important positions for the dominant Liberal Democratic Party (LDP) and Democratic Party of Japan (DPJ), two largest parties in the country at the time. Given party-level variations in the candidate evaluation, it is reasonable to assume party-specific patterns of candidate nomination. For a similar reason, standard errors are clustered at the party level. I also control for the tie dummy mentioned above, along with interaction terms between the tie dummy and the dual listing dummy, number of past wins, and incumbent dummy, where applicable. I intend to account for differential nomination patterns for those who have ties on the party list and those who do not. 

In party-specific and election year-specific analyses, I exclude corresponding fixed effects from the models. In the party-specific analysis, I analyze the four parties or party groups: LDP, DPJ and its successor Constitutional Democratic Party (CDP), Komeito, and Japan Communist Party (JCP). In the party/election-level analysis, l focus on the Liberal Democratic Party in the years 2005 and 2012.

\section{Result} \label{sec: res}

\subsection{Aggregate-Level Analysis}

I first present the result of the aggregate-level analysis, where I pool candidates across parties. Table \ref{tab:regression_results} shows the results of the regression analysis. The dependent variable for Models 1 to 4 is the candidate’s rank on the PR list. For Models 5 to 7, the dependent variable is whether the candidate is dual-listed. For Models 8 to 10, the dependent variable is whether the candidate is dual-listed and placed at the same rank as other candidates. For each dependent variable, models with and without control variables (other than fixed effects) are estimated.


\begin{table}[!htbp]
\begin{center}
\scalebox{0.6}{
\begin{threeparttable}
\begin{tabular}{l D{.}{.}{6.5} D{.}{.}{6.5} D{.}{.}{6.5} D{.}{.}{6.5} D{.}{.}{5.5} D{.}{.}{5.5} D{.}{.}{5.5} D{.}{.}{5.5} D{.}{.}{5.5} D{.}{.}{5.5}}
\toprule
 & \multicolumn{4}{c}{List Rank} & \multicolumn{3}{c}{Dual Listing} & \multicolumn{3}{c}{Dual Listing (Tie)} \\
\cmidrule(lr){2-5} \cmidrule(lr){6-8} \cmidrule(lr){9-11}
 & \multicolumn{1}{c}{Model 1} & \multicolumn{1}{c}{Model 2} & \multicolumn{1}{c}{Model 3} & \multicolumn{1}{c}{Model 4} & \multicolumn{1}{c}{Model 5} & \multicolumn{1}{c}{Model 6} & \multicolumn{1}{c}{Model 7} & \multicolumn{1}{c}{Model 8} & \multicolumn{1}{c}{Model 9} & \multicolumn{1}{c}{Model 10} \\
\midrule
Total Wins         & -0.15^{***}             &                         &                         & -0.10^{***}             & 0.16^{***}              &                         & -0.00                   & 0.14^{***}              &                         & 0.00                    \\
                   & (0.01)                  &                         &                         & (0.02)                  & (0.04)                  &                         & (0.02)                  & (0.04)                  &                         & (0.02)                  \\
Incumbency         &                         & -1.02^{***}             &                         & -0.74^{***}             &                         & 1.33^{***}              & 1.33^{***}              &                         & 1.16^{***}              & 1.13^{***}              \\
                   &                         & (0.12)                  &                         & (0.11)                  &                         & (0.28)                  & (0.28)                  &                         & (0.32)                  & (0.31)                  \\
Dual Listing       &                         &                         & -1.82^{***}             & -0.93^{***}             &                         &                         &                         &                         &                         &                         \\
                   &                         &                         & (0.25)                  & (0.27)                  &                         &                         &                         &                         &                         &                         \\
Tie                &                         &                         &                         & -1.92^{*}               &                         &                         &                         &                         &                         &                         \\
                   &                         &                         &                         & (0.86)                  &                         &                         &                         &                         &                         &                         \\
Female             &                         &                         &                         & -0.07                   &                         &                         & -0.24^{*}               &                         &                         & -0.40^{***}             \\
                   &                         &                         &                         & (0.04)                  &                         &                         & (0.12)                  &                         &                         & (0.11)                  \\
Block Magnitude    &                         &                         &                         & 0.02^{***}              &                         &                         & 0.04^{***}              &                         &                         & 0.04^{***}              \\
                   &                         &                         &                         & (0.01)                  &                         &                         & (0.01)                  &                         &                         & (0.01)                  \\
Total Wins x Tie   &                         &                         &                         & 0.10^{***}              &                         &                         &                         &                         &                         &                         \\
                   &                         &                         &                         & (0.02)                  &                         &                         &                         &                         &                         &                         \\
Tie x Incumbency   &                         &                         &                         & 0.49^{**}               &                         &                         &                         &                         &                         &                         \\
                   &                         &                         &                         & (0.18)                  &                         &                         &                         &                         &                         &                         \\
Tie x Dual Listing &                         &                         &                         & 0.84                    &                         &                         &                         &                         &                         &                         \\
                   &                         &                         &                         & (0.91)                  &                         &                         &                         &                         &                         &                         \\
\midrule
Year FE            & \multicolumn{1}{c}{Yes} & \multicolumn{1}{c}{Yes} & \multicolumn{1}{c}{Yes} & \multicolumn{1}{c}{Yes} & \multicolumn{1}{c}{Yes} & \multicolumn{1}{c}{Yes} & \multicolumn{1}{c}{Yes} & \multicolumn{1}{c}{Yes} & \multicolumn{1}{c}{Yes} & \multicolumn{1}{c}{Yes} \\
Party FE           & \multicolumn{1}{c}{Yes} & \multicolumn{1}{c}{Yes} & \multicolumn{1}{c}{Yes} & \multicolumn{1}{c}{Yes} & \multicolumn{1}{c}{Yes} & \multicolumn{1}{c}{Yes} & \multicolumn{1}{c}{Yes} & \multicolumn{1}{c}{Yes} & \multicolumn{1}{c}{Yes} & \multicolumn{1}{c}{Yes} \\
AIC                & 37167.39                & 36446.03                & 33098.42                & 31652.57                & 5886.23                 & 5697.89                 & 5629.09                 & 5949.29                 & 5803.79                 & 5729.36                 \\
Log Likelihood     & -18536.69               & -18176.02               & -16502.21               & -15771.28               & -2897.12                & -2802.94                & -2765.54                & -2928.64                & -2855.90                & -2815.68                \\
Num. obs.          & 7754                    & 7754                    & 7754                    & 7754                    & 7754                    & 7754                    & 7754                    & 7754                    & 7754                    & 7754                    \\
\bottomrule
\end{tabular}
\begin{tablenotes}[flushleft]
\scriptsize{\item $^{***}p<0.001$; $^{**}p<0.01$; $^{*}p<0.05$. Standard errors clustered at the party level in parentheses.
\item Dependent variable: candidate $i$'s list rank (columns 1-4) dual listing status (columns 5-7), and whether the candidate has a tie on the list (columns 8-10).
\item Estimated models: negatige binomial (columns 1-4) and logit (columns 5-10).}
\end{tablenotes}
\end{threeparttable}
}
\caption{Regression Results}
\label{tab:regression_results}
\end{center}
\end{table}


The models explaining PR list rank support Hypotheses \ref{hyp:first}, \ref{hyp:second}, and \ref{hyp:fifth}. Regardless of whether control variables are included, a significant relationship is observed between PR list rank and three key explanatory variables: the candidate’s past number of election victories, incumbent status, and dual-listing status. Specifically, candidates who have been elected more times in the past, are incumbents, or are dual-listed are more likely to be placed at higher positions on the PR list. Importantly, this pattern holds true for both candidates who are dual-listed with a shared rank and those who are not.

The models explaining dual-listing status support Hypothesis \ref{hyp:fourth} but not \ref{hyp:third}. While there is a significant result for the latter in the model without control variables, the result disappears once control variables are included, suggesting that the number of times a candidate has been elected does not necessarily increase the likelihood of being dual-listed. Instead, the results indicate that the key determinant of dual-listing status is incumbent status. Candidates who are incumbents are more likely to be dual-listed, suggesting that parties prioritize incumbents for dual listing.

To visualize the extent to which each explanatory variable affects the dependent variables, the marginal effects of key variables are estimated. Figure \ref{fig:marginal_rank} presents how changes in a candidate’s profile — including the number of times elected, incumbent status, and dual-listing status — affect the predicted rank on the PR list. Model 4 is used for this analysis. The hypothetical candidate is a 52-year-old male running as a Liberal Democratic Party (LDP) candidate in the Kyushu block in the 2012 general election. The district magnitude for the Kyushu block in this election was 20, which is the median district magnitude across all time periods. The figure plots the candidate’s predicted PR list rank (y-axis) against the number of previous elections won (x-axis). The solid line represents non-incumbent candidates, while the dashed line represents incumbent candidates. The red line shows the predicted rank for a candidate who is not dual-listed, the blue line represents a candidate who is dual-listed but not placed at the same rank as other candidates, and the green line represents a candidate who is dual-listed and placed at the same rank as other candidates.

\begin{figure}[!htbp]
	\includegraphics[width = 0.9\textwidth]{../figure/paper/marginal_effects_rank.pdf}
	\caption{Marginal Effects of Seniority, Incumbency, Dual Listing, and Tie Status on List Rank}
	\label{fig:marginal_rank}
\end{figure}

The most noticeable pattern is that, as shown by the green line, candidates who are dual-listed with shared ranks are consistently placed at higher PR list positions, regardless of the number of times they have been elected in the past. This pattern holds true for both incumbents and non-incumbents. This suggests that candidates who are placed at the same rank as other dual-listed candidates are given relatively high PR list positions, even if their track records are limited. However, this does not necessarily guarantee a high chance of election, since the effective rank used to allocate PR seats is adjusted downward if a candidate performs poorly in the single-member district race.

In contrast, for candidates who are not dual-listed (red line) or those who are dual-listed but not placed at a shared rank (blue line), seniority and incumbent status play a critical role. Candidates with fewer past elections and non-incumbent status are placed at significantly lower PR list positions than those who are dual-listed with shared ranks. However, candidates with a long history of electoral victories or incumbent status are predicted to receive much higher PR list positions.

From this figure, it is clear that seniority and incumbent status are crucial determinants of a candidate’s PR list position. For candidates lacking these attributes, being dual-listed becomes a critical factor in determining their chance of election via the PR system. This raises the question: to what extent do seniority and incumbent status affect the likelihood of being dual-listed? Figure \ref{marginal_dual} addresses this question.

Figure \ref{marginal_dual} examines how the likelihood of being dual-listed changes as a candidate’s number of prior elections and incumbent status change. The left panel plots the probability of being dual-listed (y-axis) against the number of past elections won (x-axis), while the right panel plots the probability of being dual-listed with a shared rank. The model used for the left panel is Model 7, and for the right panel, Model 10 is used. The hypothetical candidate profile is the same as that used for the previous figure.

\begin{figure}[!htbp]
	\includegraphics[width = 0.9\textwidth]{../figure/paper/marginal_effects_dual_tie.pdf}
	\caption{Marginal Effects of Seniority and Incumbency on Dual Listing}
	\label{fig:marginal_dual}
\end{figure}

The results indicate that while the number of times elected does not have a substantial impact on the probability of being dual-listed or being given a shared rank, incumbent status does. Incumbents are much more likely to be dual-listed and placed at shared ranks than non-incumbents. The baseline probability of being dual-listed is already high, reflecting the frequent use of dual-listing strategies by the LDP.

How should these results be interpreted? First, it is evident that parties tend to prioritize incumbent candidates for favorable treatment in the PR system. As shown in the first figure, being an incumbent provides candidates with a significant advantage in terms of PR list rank. This advantage appears to be larger than the advantage associated with seniority. Incumbents are also more likely to be dual-listed and placed at the same rank as other candidates. Since dual-listed candidates are typically assigned higher ranks on the PR list, it is clear that parties use dual listing to achieve two goals. On the one hand, parties reward incumbents who have demonstrated their capabilities in previous elections or their legislative activities. On the other hand, placing these incumbents at a shared rank with other candidates allows parties to monitor the effort they exert in the single-member district race.

Senior candidates also tend to be treated favorably, but the way they are treated differs from that of incumbents. While senior candidates benefit from high PR list ranks, they are not necessarily dual-listed. The first figure shows a clear positive relationship between seniority and PR list rank for candidates who are not dual-listed or are dual-listed without a shared rank. However, this relationship is much weaker for candidates who are dual-listed with shared ranks. In addition, the likelihood of being dual-listed with a shared rank is not higher for senior candidates than for junior candidates. This suggests that certain senior candidates receive special treatment that distinguishes them from other candidates.

Given that dual listing can be used as a mechanism to monitor candidate performance, it is plausible that senior candidates, especially those with proven experience, do not require such monitoring. Instead, they are rewarded with high PR list ranks through more direct means. This would be consistent with the treatment of influential party figures, prominent cabinet ministers, and former high-ranking officials. In these cases, the party may see little benefit in monitoring their effort in the single-member district race.

\subsection{Party-Specific Analysis}

% TODO: write
To investigate whether the trends observed so far are influenced by the presence of specific political parties and to gain insight into differences in nomination patterns across parties, an analysis at the party level was conducted. This analysis focuses on two majority-seeking parties — the dominant Liberal Democratic Party (LDP) and the Democratic Party (including its successor, the Constitutional Democratic Party (CDP)) — as well as two smaller parties, Komeito and the Japanese Communist Party (JCP), which are not considered majority-seeking. 

In the nine general elections analyzed, a total of 2,900 candidates ran in PR districts under the LDP banner. Table \ref{tab:ldp} shows the results of the regression analysis, which are consistent with Hypotheses 1, 2, 4, and 5. Candidates’ past number of electoral victories, incumbency status, and dual listing status all show a positive relationship with their placement on the PR list. While there is no clear relationship between the number of past wins and dual listing or tie status, incumbents are more likely to be granted dual listing and to be placed in tie positions compared to non-incumbents.


\begin{table}[!bth]
\begin{center}
\begin{threeparttable}
\begin{tabular}{l D{.}{.}{5.5} D{.}{.}{5.5} D{.}{.}{5.5} D{.}{.}{5.5} D{.}{.}{5.5}}
\toprule
 & \multicolumn{2}{c}{Dual Listing} & \multicolumn{3}{c}{List Rank} \\
\cmidrule(lr){2-3} \cmidrule(lr){4-6}
 & \multicolumn{1}{c}{H1} & \multicolumn{1}{c}{H2} & \multicolumn{1}{c}{H3} & \multicolumn{1}{c}{H4} & \multicolumn{1}{c}{H5} \\
\midrule
Total Wins      & 0.17^{***}              &                         & -0.14^{***}             &                         &                         \\
                & (0.02)                  &                         & (0.01)                  &                         &                         \\
Incumbency      &                         & 1.59^{***}              &                         & -1.22^{***}             &                         \\
                &                         & (0.12)                  &                         & (0.04)                  &                         \\
Dual Listing    &                         &                         &                         &                         & -1.96^{***}             \\
                &                         &                         &                         &                         & (0.04)                  \\
Female          & -0.58^{**}              & -0.56^{**}              & -0.01                   & -0.05                   & -0.09                   \\
                & (0.19)                  & (0.19)                  & (0.08)                  & (0.08)                  & (0.06)                  \\
Block Magnitude & 0.05^{***}              & 0.06^{***}              & 0.00                    & 0.00                    & 0.01^{***}              \\
                & (0.01)                  & (0.01)                  & (0.00)                  & (0.00)                  & (0.00)                  \\
\midrule
Year FE         & \multicolumn{1}{c}{Yes} & \multicolumn{1}{c}{Yes} & \multicolumn{1}{c}{Yes} & \multicolumn{1}{c}{Yes} & \multicolumn{1}{c}{Yes} \\
Party FE        & \multicolumn{1}{c}{Yes} & \multicolumn{1}{c}{Yes} & \multicolumn{1}{c}{Yes} & \multicolumn{1}{c}{Yes} & \multicolumn{1}{c}{Yes} \\
AIC             & 2312.75                 & 2192.52                 & 14665.40                & 14239.73                & 12869.00                \\
BIC             & 2377.20                 & 2256.97                 & 14735.71                & 14310.04                & 12939.31                \\
Log Likelihood  & -1145.37                & -1085.26                & -7320.70                & -7107.87                & -6422.50                \\
Deviance        & 2290.75                 & 2170.52                 & 2609.56                 & 2526.28                 & 2149.91                 \\
Num. obs.       & 2589                    & 2589                    & 2589                    & 2589                    & 2589                    \\
\bottomrule
\end{tabular}
\begin{tablenotes}[flushleft]
\scriptsize{\item $^{***}p<0.001$; $^{**}p<0.01$; $^{*}p<0.05$. Standard errors in parentheses.
\item Dependent variable: candidate $i$'s list rank (H1-2) and dual nomination status (H3-5).
\item Estimated models: logit (H1-2) and negative binomial (H3-5).}
\end{tablenotes}
\end{threeparttable}
\caption{Regression Results for LDP Candidates}
\label{tab:regLDP}
\end{center}
\end{table}


Table \ref{tab:dpj_cdp} shows the nomination patterns of the DPJ and CDP. In the nine general elections analyzed, a total of 2,093 candidates ran for PR seats under the DPJ and CDP. Unlike the aggregated analysis and the LDP analysis, there is no clear relationship between the number of past wins and list rank in models that control for other factors (Model 4). However, incumbents and dual candidates tend to receive higher list ranks, supporting Hypotheses 2 and 5. Additionally, unlike in the LDP analysis, a positive relationship is observed between the number of past wins/incumbency status and dual listing. Overall, the nomination patterns differ from those of the LDP. The DPJ - CDP appears to prioritize incumbency status rather than past electoral victories when determining candidate treatment.


\begin{table}[!htbp]
\begin{center}
\scalebox{0.6}{
\begin{threeparttable}
\begin{tabular}{l D{.}{.}{5.5} D{.}{.}{5.5} D{.}{.}{5.5} D{.}{.}{5.5} D{.}{.}{4.5} D{.}{.}{4.5} D{.}{.}{4.5} D{.}{.}{4.5} D{.}{.}{4.5} D{.}{.}{4.4}}
\toprule
 & \multicolumn{4}{c}{List Rank} & \multicolumn{3}{c}{Dual Listing} & \multicolumn{3}{c}{Dual Listing (Tie)} \\
\cmidrule(lr){2-5} \cmidrule(lr){6-8} \cmidrule(lr){9-11}
 & \multicolumn{1}{c}{Model 1} & \multicolumn{1}{c}{Model 2} & \multicolumn{1}{c}{Model 3} & \multicolumn{1}{c}{Model 4} & \multicolumn{1}{c}{Model 5} & \multicolumn{1}{c}{Model 6} & \multicolumn{1}{c}{Model 7} & \multicolumn{1}{c}{Model 8} & \multicolumn{1}{c}{Model 9} & \multicolumn{1}{c}{Model 10} \\
\midrule
Total Wins       & -0.15^{***}             &                         &                         & 0.04                    & 0.38^{***}              &                         & 0.18^{*}                & 0.30^{***}              &                         & 0.15^{*}                \\
                 & (0.01)                  &                         &                         & (0.04)                  & (0.07)                  &                         & (0.07)                  & (0.06)                  &                         & (0.07)                  \\
Incumbency       &                         & -0.84^{***}             &                         & -1.46^{***}             &                         & 1.50^{***}              & 1.05^{***}              &                         & 1.17^{***}              & 0.77^{**}               \\
                 &                         & (0.05)                  &                         & (0.15)                  &                         & (0.23)                  & (0.28)                  &                         & (0.20)                  & (0.25)                  \\
Dual Listing     &                         &                         & -2.61^{***}             & -2.38^{***}             &                         &                         &                         &                         &                         &                         \\
                 &                         &                         & (0.05)                  & (0.27)                  &                         &                         &                         &                         &                         &                         \\
Tie              &                         &                         &                         & -0.30                   &                         &                         &                         &                         &                         &                         \\
                 &                         &                         &                         & (0.27)                  &                         &                         &                         &                         &                         &                         \\
Female           &                         &                         &                         & -0.06                   &                         &                         & -0.07                   &                         &                         & -0.19                   \\
                 &                         &                         &                         & (0.05)                  &                         &                         & (0.21)                  &                         &                         & (0.20)                  \\
Block Magnitude  &                         &                         &                         & 0.02^{***}              &                         &                         & 0.04^{**}               &                         &                         & 0.03^{**}               \\
                 &                         &                         &                         & (0.00)                  &                         &                         & (0.01)                  &                         &                         & (0.01)                  \\
Total Wins x Tie &                         &                         &                         & -0.05                   &                         &                         &                         &                         &                         &                         \\
                 &                         &                         &                         & (0.04)                  &                         &                         &                         &                         &                         &                         \\
Tie x Incumbency &                         &                         &                         & 1.34^{***}              &                         &                         &                         &                         &                         &                         \\
                 &                         &                         &                         & (0.16)                  &                         &                         &                         &                         &                         &                         \\
\midrule
Year FE          & \multicolumn{1}{c}{Yes} & \multicolumn{1}{c}{Yes} & \multicolumn{1}{c}{Yes} & \multicolumn{1}{c}{Yes} & \multicolumn{1}{c}{Yes} & \multicolumn{1}{c}{Yes} & \multicolumn{1}{c}{Yes} & \multicolumn{1}{c}{Yes} & \multicolumn{1}{c}{Yes} & \multicolumn{1}{c}{Yes} \\
Party FE         & \multicolumn{1}{c}{No}  & \multicolumn{1}{c}{No}  & \multicolumn{1}{c}{No}  & \multicolumn{1}{c}{No}  & \multicolumn{1}{c}{No}  & \multicolumn{1}{c}{No}  & \multicolumn{1}{c}{No}  & \multicolumn{1}{c}{No}  & \multicolumn{1}{c}{No}  & \multicolumn{1}{c}{No}  \\
AIC              & 9118.31                 & 9030.19                 & 7149.22                 & 6995.05                 & 1106.75                 & 1101.45                 & 1087.57                 & 1177.03                 & 1175.42                 & 1165.20                 \\
BIC              & 9180.42                 & 9092.30                 & 7211.33                 & 7096.68                 & 1163.21                 & 1157.91                 & 1160.97                 & 1233.49                 & 1231.88                 & 1238.60                 \\
Log Likelihood   & -4548.16                & -4504.10                & -3563.61                & -3479.53                & -543.37                 & -540.72                 & -530.79                 & -578.51                 & -577.71                 & -569.60                 \\
Deviance         & 1838.19                 & 1806.51                 & 1115.73                 & 1105.82                 & 1086.75                 & 1081.45                 & 1061.57                 & 1157.03                 & 1155.42                 & 1139.20                 \\
Num. obs.        & 2093                    & 2093                    & 2093                    & 2093                    & 2093                    & 2093                    & 2093                    & 2093                    & 2093                    & 2093                    \\
\bottomrule
\end{tabular}
\begin{tablenotes}[flushleft]
\scriptsize{\item $^{***}p<0.001$; $^{**}p<0.01$; $^{*}p<0.05$. Standard errors in parentheses.
\item Dependent variable: candidate $i$'s list rank (columns 1-4) dual listing status (columns 5-7), and whether the candidate has a tie on the list (columns 8-10).
\item Estimated models: negatige binomial (columns 1-4) and logit (columns 5-10).}
\end{tablenotes}
\end{threeparttable}
}
\caption{Regression Results for DPJ / CDP Candidates}
\label{tab:dpj_cdp}
\end{center}
\end{table}


During the nine general elections, Komeito fielded a total of 325 candidates in the PR districts. Table \ref{tab:komeito} presents the analysis results. Compared to the LDP and the DPJ-CDP, the number of Komeito candidates is significantly smaller. Komeito is known for its policy of not employing dual listing, and this is reflected in the data: no relationship is observed between dual listing status and other variables. However, as in the other parties, senior candidates and incumbents are placed in higher positions on the PR list. Unlike the LDP, which relies on dual listing as a tool to manage candidates, Komeito seems to prioritize its senior and incumbent candidates by placing them directly in higher list positions.

Similarly, the JCP nominated a total of 466 candidates ran in PR districts during the nine general elections, a number that is only slightly higher than that of Komeito. Table \ref{tab:jcp} shows the analysis results. Similar to Komeito, the JCP does not appear to rely on dual listing as a means of favoring senior or incumbent candidates. However, as with Komeito, senior candidates and incumbents are placed higher on the PR list. This suggests that, for both Komeito and the JCP, senior and incumbent candidates are treated favorably, not through dual listing, but rather through direct allocation to higher list positions.

Across all parties, senior candidates and incumbents tend to be placed in higher positions on the PR list. However, the methods used to provide preferential treatment vary by party. Of the four parties analyzed, the LDP tends to offer dual listing opportunities to incumbent candidates, but not necessarily to senior candidates. In contrast, the DPJ - CDP is more likely to offer dual listing to both senior and incumbent candidates. Komeito and the JCP, which generally do not use dual listing, instead offer more direct preferential treatment by placing such candidates in safe PR list positions. This difference in strategy suggests that while all parties have an incentive to prioritize experienced candidates, the mechanism they employ — whether through dual listing or direct list placement — varies according to party strategy and organizational norms.

\subsection{Election- / Party-Specific Analysis}

Next, we examine the applicability of the hypotheses for specific parties during specific elections. In particular, it is expected that Hypotheses 1 and 3 will be less applicable to parties that have recently lost power or experienced internal conflicts. Specifically, we focus on the LDP during the 2005 and 2012 general elections.

The House of Representatives election in September 2005, widely known as the "Yusei Senkyo (Postal Election)," was precipitated by the dissolution of the legislature following the rejection of the Postal Privatization Bill in the House of Councillors the preceding month. For Prime Minister Junichiro Koizumi, the bill represented a central policy initiative that he had championed even before his tenure as Prime Minister \citep{uchiyamaKoizumiSeikenKoizumis2007}. However, the legislative process was marked by significant resistance from within the party. In an unprecedented move, Koizumi circumvented the conventional pre-legislative scrutiny within the party to bring the bill directly before the plenary session of the House of Representatives.\footnotemark{} Although the bill narrowly passed in the House of Representatives, it was ultimately rejected by the House of Councillors. Throughout this process, a total of 51 LDP members in the House of Representatives and 22 members in the House of Councillors dissented by voting against the bill or absenting from the session. 

\footnotetext{In the LDP government, it is standard procedure for bills proposed by the cabinet to undergo preliminary review by the LDP's Soumukai (General Council) before being presented to the Diet. Party regulations stipulate that decisions within the General Council are typically made by a majority vote. However, the principle of achieving unanimous agreement is commonly upheld as a procedural norm. The Postal Privatization Bill, which was subject to this review process, advanced through the General Council by securing a majority vote, despite the usual emphasis on achieving consensus.}

Following the rejection of the bill, Koizumi dissolved the House of Representatives and called a general election. During this election, he revoked the official endorsement of the members who had rebelled during the bill's vote and placed other officially endorsed candidates in their constituencies. The displaced members either contested the election as independents or established new political parties to do so. The Liberal Democratic Party (LDP) secured a landslide victory in this election, followed by the swift passage of the once-denied Postal Privatization Bill through both chambers of the Diet shortly thereafter. This turnover of members might have resulted in changed to the candidate nomination patterns in the PR tier. 

In the 2009 general election, the LDP suffered a significant defeat, losing its majority and being ousted from power.  The DPJ won 308 of the 480 available seats, while the LDP’s representation plummeted from 300 seats before the election to just 119. However, in the 2012 election, the LDP staged a comeback, winning 294 seats and regaining power. The turnover in members might have also led to changes in the candidate nomination patterns. 

Table \ref{tab:ldp_2005_2012} presents the analysis of LDP candidate nomination patterns for the 2005 and 2012 elections. Notably, in 2012, all dual-listed candidates were placed at the same position within each list, so the tie dummy is excluded.


\begin{table}[!bth]
\begin{center}
\begin{threeparttable}
\begin{tabular}{l D{.}{.}{4.5} D{.}{.}{4.5} D{.}{.}{4.5} D{.}{.}{4.5}}
\toprule
 & \multicolumn{2}{c}{2005 LDP} & \multicolumn{2}{c}{2012 LDP} \\
\cmidrule(lr){2-3} \cmidrule(lr){4-5}
 & \multicolumn{1}{c}{H1} & \multicolumn{1}{c}{H3} & \multicolumn{1}{c}{H1} & \multicolumn{1}{c}{H3} \\
\midrule
Total Wins      & 0.30^{***} & -0.12^{***} & 0.83^{***} & -0.28^{***} \\
                & (0.08)     & (0.02)      & (0.21)     & (0.03)      \\
Female          & 0.34       & -0.77^{***} & 0.59       & -0.17       \\
                & (0.58)     & (0.18)      & (0.68)     & (0.27)      \\
Block Magnitude & 0.04       & 0.02^{***}  & 0.10^{***} & -0.03^{*}   \\
                & (0.02)     & (0.01)      & (0.03)     & (0.01)      \\
\midrule
AIC             & 290.29     & 1966.49     & 227.62     & 1622.67     \\
BIC             & 305.56     & 1985.57     & 242.77     & 1641.60     \\
Log Likelihood  & -141.14    & -978.24     & -109.81    & -806.33     \\
Deviance        & 282.29     & 326.21      & 219.62     & 322.75      \\
Num. obs.       & 336        & 336         & 326        & 326         \\
\bottomrule
\end{tabular}
\begin{tablenotes}[flushleft]
\scriptsize{\item $^{***}p<0.001$; $^{**}p<0.01$; $^{*}p<0.05$. Standard errors in parentheses.
\item Dependent variable: candidate $i$'s dual listing status (H1) and list rank (H3).
\item Estimated models: logit (H1) and negative binomial (H3).}
\end{tablenotes}
\end{threeparttable}
\caption{Regression Results for LDP Candidates in 2005 and 2012}
\label{tab:regLDP2005_2012}
\end{center}
\end{table}


As expected, Hypotheses 1 and 3 do not hold for either the 2005 or 2012 elections. In the 2005 election, incumbents were given more favorable PR list placements, but this pattern disappeared in 2012. The 2009 electoral defeat and subsequent turnover of LDP candidates likely played a role.

What about dual listing? In the 2005 election, incumbents were more likely to be given dual listing opportunities, a pattern consistent with the overall LDP analysis. However, there was no clear relationship between the number of past wins and dual listing. In 2012, candidates with more past wins were more likely to be given dual listing opportunities. The events of 2005 — where a large number of incumbents and senior members were forced to run as independents or were replaced by newcomers — likely influenced this pattern.

The combined results of the 2005 and 2012 election analyses suggest that party-level disruptions, such as internal conflicts or the experience of losing power, weaken the explanatory power of the hypotheses. In the 2005 “Postal Election,” the LDP’s prioritization of incumbents in the PR list rankings is observed, but in 2012, this pattern disappears. Additionally, while seniority and incumbency status are typically strong predictors of dual listing, the results are less clear for the 2005 and 2012 elections. This highlights how political upheaval, party reorganization, and leadership decisions can shift candidate nomination patterns.

\section{Discussion} \label{sec: dis}

The analysis thus far generally supports the hypotheses presented in this paper. Specifically, under Japan’s interactive MMM system, political parties give preferential treatment to candidates with multiple election victories or incumbents by placing them higher on party lists or enabling them to run under dual listing. While each party has its own unique patterns of candidate selection, all parties exhibit a tendency to prioritize senior and incumbent candidates in some form. However, in cases where parties experience internal conflict or when a ruling party is ousted from power, turnover tends to increase, and the preferential treatment of existing candidates weakens.

The observed trends indicate that interactive MMM reduces the likelihood of candidate turnover. This, in turn, suggests that the PR tier of interactive MMM does not necessarily promote minority representation in the same way that a standard PR system would.

\subsection{Legislative Turnover}

The regular turnover of candidates and legislators is crucial for political parties \citep{matlandDeterminantsLegislativeTurnover2004}. Turnover enables parties to discover future political leaders and offer opportunities to ambitious individuals. To some extent, internal party turnover signifies healthy competition among party members.

Dual listing is believed to reduce the frequency of turnover. Parties forecast the number of seats they can realistically win and then select candidates accordingly. Since the number of candidates that a party can field is limited, they distribute these slots to candidates based on party priorities. Dual listing allows parties to offer “insurance tickets” to specific candidates in a selective manner. Parties do not merely seek to maximize votes; they also pursue office-seeking and policy-seeking objectives. This creates incentives for parties to prioritize senior and incumbent candidates. As the analysis has shown, parties do indeed use insurance tickets. Under these conditions, the total number of candidates fielded by a party is often lower than it would be if dual listing were not allowed \citep[p.448]{mckeanJapansNewElectoral2000}. With dual listing, one candidate is effectively “placed” in two slots, reducing the number of new candidates that might have otherwise been fielded. Additionally, even if such new candidates are fielded, they may fail to win due to the presence of “revived” candidates (those elected via the PR list after losing in the single-member district) occupying seats.

This paper does not directly conduct a causal analysis of how much dual listing increases a candidate’s chance of winning. However, as evidenced by the support for Hypothesis 1, parties have a tendency to rank dual-listed candidates higher on the party list. Given the well-established relationship between list ranking and electability in closed-list PR systems, it is reasonable to infer that dual-listed candidates have a higher probability of being elected than candidates who are not dual listed.\footnotemark{} 

\footnotetext{For example, \citet{manowElectoralRulesLegislative2007} demonstrated that dual-listed candidates in Germany have a higher probability of re-election.}

\subsection{Representation}

It is well established that PR electoral systems are generally better suited than majoritarian systems for promoting minority representation \citep{matlandContagionWomenCandidates1996, matlandWomensRepresentationNational1998, meserveGenderIncumbencyParty2020}. 

The initial motivation for adopting PR systems was to provide representation for the diverse interests of voters. As the name suggests, PR systems exhibit a stronger proportional relationship between votes and seats. Unlike in majoritarian systems, smaller parties with a relatively small share of the vote have a higher chance of winning seats under PR. This allows parties that would not have secured representation under a majoritarian system to enter the legislature, thereby reflecting the interests of minority groups in policymaking. Historically, the introduction of PR systems before World War I was driven, in part, by a desire to promote the parliamentary representation of minority groups within domestic political communities \citep{rokkanCitizensElectionsParties1970}.

However, the introduction of PR is not always driven by a desire to promote minority representation \citep{ahmedReadingHistoryForward2010, boixSettingRulesGame1999, cusackEconomicInterestsOrigins2007}. Rather, the enhanced representation of minority groups under PR results from the interaction between voters and parties. Under PR, voters evaluate party lists collectively, unlike in majoritarian systems where voters assess individual candidates. Given this context, parties have an incentive to present diverse lists of candidates to appeal to a broader range of voters. Large parties, in particular, may include candidates who represent minority interests as a way to attract a wider electorate \citep[p.188]{norrisElectoralEngineeringVoting2004}.

However, when PR is embedded within a mixed-member system, the representational advantage of PR does not necessarily persist. In interactive MMM systems, the incentives for parties to field a diverse range of candidates on PR lists are diminished by the competing incentive to offer insurance tickets to senior and incumbent candidates. Unless large parties have strong incentives to appeal to a broad base of voters in the PR tier, they are likely to prioritize senior and incumbent candidates for dual listing or place them higher on the list. As a result, minority representation can be suppressed.

I illustrate the above discussion with the example of youth underrepresentation in Japan. Japan's House of Represetatives severely underrepresent young citizens: it has only six percent of legislators under 40 as of 2023, the fewest among G7 countries (Table \ref{table:intl}) and third-fewest among 37 OECD countries \citep{mccleanSilverDemocracyYouth2020}. What is noteworthy is that, despite Japan’s use of both single-member districts and PR, there is no observable age difference between candidates in the two tiers (Figure \ref{fig:pr_vs_smd}).

\begin{table}[htbp]
\begin{center}
\begin{threeparttable}
\input{../table/tab.age.intl.tex}
\begin{tablenotes}[flushleft]
  \scriptsize{
    \item{\textit{Note.} Age demographics of lower house members in the G7 countries, as of January 2023. Eligibility is the minimum age to run for the house.}
    \item{\textit{Source.} \citet{inter-parliamentaryunionDataAgeCountry2024}.}
  }
\end{tablenotes}
\end{threeparttable}
\caption{Age Demographics of Lower Houses in the G7 Countries}
\label{table:intl}
\end{center}
\end{table}

\begin{figure}[!htbp]
	\includegraphics[width = 0.9\textwidth]{../figure/paper/age_smd_vs_pr_winners.pdf}
	\caption{Age Composition of Legislators Elected from the Two Tiers}
	\label{fig:pr_vs_smd}
\end{figure}


The degree to which young people are descriptively represented in parliament depends on two factors: (1) the extent to which new legislators enter parliament and (2) the age at which they enter. This is because first-time legislators are typically younger than the parliamentary average. Table \ref{tab:firstElection} shows that, in every election since World War II, first-time legislators have been younger than the overall average. In recent years, newly elected legislators have been, on average, 5-10 years younger than the rest of the legislature.

% created manually, based on prepCareer.Rmd
% for each general election, display: 
% mean age of those elected for the first time
% number of those elected for the first time
% proportion of such candidates

\begin{table}[ht]
\begin{threeparttable}
\begin{tabular}{c|ccccccccc}
\toprule
Year & 1947 & 1949 & 1952 & 1953 & 1955 & 1958 & 1960 & 1963 & 1967 \\
\midrule
Mean age & 48.7 & 47.4 & 54.6 & 52.3 & 52.2 & 49.0 & 48.4 & 47.1 & 46.1 \\
Proportion (\%) & 100 & 47.4 & 43.8 & 14.6 & 16.1 & 15.0 & 13.1 & 14.6 & 21.0 \\
Mean age (all) & 48.7 & 48.6 & 52.8 & 52.6 & 53.9 & 54.6 & 55.6 & 56.1 & 56.2 \\
\midrule 
Year & 1969 & 1972 & 1976 & 1979 & 1980 & 1983 & 1986 & 1990 & 1993 \\
\midrule 
Mean age & 45.3 & 47.8 & 48.0 & 49.0 & 45.2 & 48.7 & 48.4 & 48.9 & 44.1 \\
Proportion (\%) & 19.3 & 18.9 & 24.3 & 14.5 & 6.9 & 16.4 & 12.7 & 26.0 & 26.2 \\
Mean age (all) & 55.1 & 55.3 & 55.0 & 55.8 & 56.1 & 56.0 & 56.9 & 56.4 & 54.3 \\
\midrule 
Year & 1996 & 2000 & 2003 & 2005 & 2009 & 2012 & 2014 & 2017 & 2021 \\
\midrule 
Mean age & 48.7 & 46.4 & 44.5 & 44.4 & 46.3 & 44.8 & 47.2 & 47.7 & 50.3 \\
Proportion (\%) & 23.0 & 22.1 & 20.8 & 21.0 & 32.9 & 38.3 & 9.1 & 12.0 & 8.6 \\
Mean age (all) & 55.2 & 54.6 & 53.1 & 52.4 & 52.2 & 51.9 & 53.0 & 54.7 & 55.5 \\
\bottomrule
\end{tabular}
\begin{tablenotes}[flushleft]
  \scriptsize{
    \item \textit{Note}: Mean age and proportion of MPs elected for the first time, and mean age of All MPs elected in each general election. 
    \item \textit{Data source}: Reed and Smith (2017)
  }
\end{tablenotes}
\end{threeparttable}
\caption{Data of First-Time Winners}
\label{tab:firstElection}
\end{table}

Part of Japan’s youth underrepresentation problem can be attributed to dual listing. If dual listing were prohibited, parties would have been compelled to field additional candidates to fill the slots that are currently occupied by dual-listed candidates. Given that first-time candidates are generally younger than the overall candidate pool, the prohibition of dual listing would have likely contributed to a rejuvenation of the legislature. Figure \ref{fig:ageFirstRun} below shows that from 1996 to 2017, first-time candidates were, on average, about 3 years younger than the overall pool of candidates. The same trend is observed when comparing the average age of PR candidates and first-time PR candidates (Figure \ref{fig:ageFirstWin}).

\begin{figure}[!htbp]
	\includegraphics[width = 0.9\textwidth]{../figure/paper/age_first_run.pdf}
	\caption{Age Comparison: Average vs. New Candidates}
	\label{fig:ageFirstRun}
\end{figure}

If dual listing were prohibited, parties would have needed to field new candidates, many of whom would likely have been younger. This would have partially mitigated youth underrepresentation. Moreover, the dual listing system grants parties an institutionalized mechanism for protecting senior and incumbent candidates, thereby slowing turnover and reducing the opportunities for younger candidates to be elected.

Extending the discussion beyond youth representation, the candidate selection patterns observed under interactive MMM suggest that it preserves representational imbalances that existed at the time of the system’s introduction. Dual listing privileges existing candidates and reduces the number of new candidates fielded. This arrangement has implications for the representation of other underrepresented groups, such as women and religious/ethnic minorities in other contexts. In Japan, the electoral reform of 1994 may have unintentionally preserved the representational imbalances that existed at that time. Electoral institutions, once introduced, can have lasting negative effects on the representation of minority groups. This underscores the long-term consequences of electoral system design, even when such effects were unintended.

\section{Conclusion}

This paper analyzed party candidate nominations in the PR tier of mixed-member majoritarian systems to clarify how the typical minority-promoting effects of proportional representation change under a mixed electoral system. Specifically, using the case of Japan’s House of Representatives elections, the paper demonstrated that in interactive MMM — where parties can nominate the same candidates in both single-member districts (SMD) and PR lists — parties have an incentive to favor senior and incumbent candidates by dual listing them and placing them higher on the PR list. This tendency reflects the fact that most parties are not solely focused on maximizing votes. Instead, they also pursue post-election goals such as policy implementation, legislative management, and party organization maintenance. To achieve these objectives, it is crucial not only to increase the total number of elected representatives but also to ensure the continued election of experienced politicians with greater resources. Dual listing provides parties with a tool to achieve this goal. Moreover, by using dual listing, parties can prevent candidates running in PR from slacking off in their SMD campaigns. From the perspective of candidates, having two chances to win (via SMD and PR) is also an attractive prospect, while they would be less concerned about nepotism.

The analysis revealed that senior and incumbent candidates are more likely to be placed at the top of PR lists in interactive MMM, and incumbents are particularly likely to be dual listed. While this tendency is observed across parties, it appears to weaken during periods of candidate turnover. For example, in the case of the Liberal Democratic Party, the preferential treatment of senior and incumbent candidates diminished during moments of internal conflict, such as the 2005 election, and following a change of government, as seen in the 2012 election.

The findings of this paper provide insights into a key concept in comparative politics: the representational advantage of proportional representation relative to majoritarian systems. Specifically, the analysis suggests that if proportional representation is embedded within a mixed system without careful institutional design, it may behave more like a majoritarian system. In interactive MMM, party preferences for existing candidates reduce the likelihood of new candidates being elected and limit the total number of candidates a party nominates. If the frequency or level of candidate turnover declines, any representational imbalances among social groups present at a given point in time may persist for a prolonged period. In other words, inadvertent institutional design may unintentionally sustain long-term disadvantages for certain groups.

Future research should analyze candidate nomination patterns in other countries that use MMM. While 36 countries employ some form of a mixed electoral system, 27 of them use MMM. Despite its widespread use, to the best of the author’s knowledge, there has been no prior research on candidate nomination strategies in MMM systems. This paper focused on interactive MMM, a specific subtype of MMM, but it would be of interest to investigate whether similar conclusions hold in systems where no interaction exists between the two electoral tiers. If such conclusions are found, it would offer the critical insight that proportional representation within MMM generally behaves like a majoritarian system.

Another point of interest is how outcomes differ in mixed systems that employ an open-list PR system instead of a closed-list system like Japan’s. Unlike closed-list PR, open-list PR places greater emphasis on candidates’ ability to secure personal votes \citep{nemotoLocalismCoordinationThree2013, shugartLookingLocalsVoter2005}. In such cases, it is likely that parties will have a stronger incentive to diversify the portfolio of candidates on their PR lists, which would create a more pronounced trade-off with the incentive to favor incumbent candidates.

\newpage

\bibliography{../bibliography.bib}
\bibliographystyle{apalike}

\newpage

\appendix

\setcounter{table}{0}
\setcounter{figure}{0}
\renewcommand{\thetable}{A\arabic{table}}
\renewcommand{\thefigure}{A\arabic{figure}}

\section{Summary Statistics}

\subsection{Candidate-Level Summary Statistics}

% TODO: add a revised table
\begin{table}[!htbp] \centering \renewcommand*{\arraystretch}{1.1}\caption{Summary Statistics}\resizebox{\textwidth}{!}{
\begin{threeparttable}
\begin{tabular}{lcccccccc}
\toprule
\multirow{2}{*}{Year} & \multirow{2}{*}{N} & \multicolumn{4}{c}{Proportion (\%)} & \multicolumn{3}{c}{N of Wins} \\
\cmidrule(lr){3-6} \cmidrule(lr){7-9}
 &  & Incumbents & Dual-Listed & Tie & Female & Mean & SD & Median\\ 
\midrule
1996 & 809 & 38.2 & 70.1 & 62.1 & 9.3 & 1.78 & 2.90 & 0 \\
2000 & 904 & 43.8 & 77.3 & 70.7 & 11.4 & 1.71 & 2.59 & 1 \\
2003 & 745 & 49.3 & 82.1 & 76.8 & 10.3 & 1.70 & 2.27 & 1 \\
2005 & 778 & 51.9 & 81.7 & 76.1 & 10.8 & 1.95 & 2.49 & 1 \\
2009 & 887 & 47.1 & 73.5 & 70.5 & 14.4 & 1.84 & 2.48 & 1 \\
2012 & 1117 & 37.2 & 81.2 & 79.3 & 12.9 & 1.41 & 2.16 & 0 \\
2014 & 841 & 51.4 & 72.4 & 69.1 & 14.9 & 2 & 2.51 & 1 \\
2017 & 855 & 45.4 & 71.5 & 67.5 & 17.0 & 1.96 & 2.47 & 1\\
2021 & 818 & 46.3 & 76.3 & 72.2 & 17.4 & 1.59 & 2.52 & 0 \\
\midrule
Total & 7754 & 45.3 & 76.3 & 71.8 & 13.2 & 1.76 & 2.49 & 1 \\
\bottomrule
\end{tabular}
\begin{tablenotes}[flushleft]
  \scriptsize{
    \item \textit{Note}: Summary statistics for the candidates who ran in the PR tier of each general election.
    \item \textit{Data source}: \citet{reedReedSmithJapaneseHouse2017} 
  }
\end{tablenotes}
\end{threeparttable}
}
\label{tab:stats}
\end{table}



\newpage

\subsection{Magnitudes of PR Blocks, 1996 - 2021}

% created manually
\begin{table}[!htbp]
\begin{threeparttable}
\begin{tabular}{lccccccccc}
\toprule
Bloc & 1996 & 2000 & 2003 & 2005 & 2009 & 2012 & 2014 & 2017 & 2021 \\
\midrule
Hokkaido & 9 & 8 & 8 & 8 & 8 & 8 & 8 & 8 & 8 \\
Tohoku & 16 & 14 & 14 & 14 & 14 & 14 & 14 & 13 & 13 \\
Kita-kanto & 21 & 20 & 20 & 20 & 20 & 20 & 20 & 19 & 19 \\
Tokyo & 19 & 17 & 17 & 17 & 17 & 17 & 17 & 17 & 17 \\
Minami-kanto & 23 & 21 & 22 & 22 & 22 & 22 & 22 & 22 & 22 \\
Hokuriku Shinetsu & 13 & 11 & 11 & 11 & 11 & 11 & 11 & 11 & 11 \\
Tokai & 23 & 21 & 21 & 21 & 21 & 21 & 21 & 21 & 21 \\
Kinki & 33 & 30 & 30 & 30 & 29 & 29 & 29 & 28 & 28 \\
Chugoku & 13 & 11 & 11 & 11 & 11 & 11 & 11 & 11 & 11 \\
Shikoku & 7 & 6 & 6 & 6 & 6 & 6 & 6 & 6 & 6 \\
Kyushu & 23 & 21 & 21 & 21 & 21 & 21 & 21 & 20 & 20 \\
\bottomrule
\end{tabular}
\begin{tablenotes}[flushleft]
  \scriptsize{
    \item \textit{Note}: Magnitudes of each PR regional district for elections 1996 - 2021. 
    \item \textit{Data source}: \citet{reedReedSmithJapaneseHouse2017, ministryofinternalaffairsandcommunicationsElectionSenkyo2024}
  }
\end{tablenotes}
\end{threeparttable}
\caption{Magnitudes of PR Blocks}
\label{tab:distM}
\end{table}

\newpage

\subsection{Distribution of List Rank}

\begin{figure}[!htbp]
	\includegraphics[width = 0.9\textwidth]{../figure/paper/pr_rank_distribution.pdf}
	\caption{Distribution of List Rank}
	\label{fig:distRank}
\end{figure}

\newpage

\subsection{Age of Winners}

\begin{figure}[!htbp]
	\includegraphics[width = 0.9\textwidth]{../figure/paper/age_first_win.pdf}
	\caption{Age Comparison: Average vs. New Legislators}
	\label{fig:ageFirstWin}
\end{figure}	

\newpage

\section{Additional Analyses}

\subsection{Party-level Analysis}

\subsubsection*{Komeito}


\begin{table}[!htbp]
\begin{center}
\scalebox{0.7}{
\begin{threeparttable}
\begin{tabular}{l D{.}{.}{3.5} D{.}{.}{4.5} D{.}{.}{4.3} D{.}{.}{4.5} D{.}{.}{3.3} D{.}{.}{3.3} D{.}{.}{3.3} D{.}{.}{3.3} D{.}{.}{3.3} D{.}{.}{3.3}}
\toprule
 & \multicolumn{4}{c}{List Rank} & \multicolumn{3}{c}{Dual Listing} & \multicolumn{3}{c}{Dual Listing (Tie)} \\
\cmidrule(lr){2-5} \cmidrule(lr){6-8} \cmidrule(lr){9-11}
 & \multicolumn{1}{c}{Model 1} & \multicolumn{1}{c}{Model 2} & \multicolumn{1}{c}{Model 3} & \multicolumn{1}{c}{Model 4} & \multicolumn{1}{c}{Model 5} & \multicolumn{1}{c}{Model 6} & \multicolumn{1}{c}{Model 7} & \multicolumn{1}{c}{Model 8} & \multicolumn{1}{c}{Model 9} & \multicolumn{1}{c}{Model 10} \\
\midrule
Total Wins       & -0.21^{***}             &                         &                         & -0.12^{***}             & 0.32                    &                         & 0.12                    & 0.34                    &                         & 0.38                    \\
                 & (0.02)                  &                         &                         & (0.03)                  & (0.19)                  &                         & (0.27)                  & (0.24)                  &                         & (0.34)                  \\
Incumbency       &                         & -0.79^{***}             &                         & -0.46^{***}             &                         & 1.50                    & 1.21                    &                         & 1.17                    & 0.35                    \\
                 &                         & (0.07)                  &                         & (0.12)                  &                         & (0.88)                  & (1.15)                  &                         & (1.25)                  & (1.70)                  \\
Dual Listing     &                         &                         & -0.26                   & 0.20                    &                         &                         &                         &                         &                         &                         \\
                 &                         &                         & (0.23)                  & (0.30)                  &                         &                         &                         &                         &                         &                         \\
Tie              &                         &                         &                         & -1.63                   &                         &                         &                         &                         &                         &                         \\
                 &                         &                         &                         & (1.96)                  &                         &                         &                         &                         &                         &                         \\
Female           &                         &                         &                         & -0.13                   &                         &                         & -0.49                   &                         &                         & 1.06                    \\
                 &                         &                         &                         & (0.10)                  &                         &                         & (1.19)                  &                         &                         & (1.45)                  \\
Block Magnitude  &                         &                         &                         & 0.04^{***}              &                         &                         & -0.06                   &                         &                         & -0.05                   \\
                 &                         &                         &                         & (0.01)                  &                         &                         & (0.06)                  &                         &                         & (0.10)                  \\
Total Wins x Tie &                         &                         &                         & 0.64                    &                         &                         &                         &                         &                         &                         \\
                 &                         &                         &                         & (0.71)                  &                         &                         &                         &                         &                         &                         \\
\midrule
Year FE          & \multicolumn{1}{c}{Yes} & \multicolumn{1}{c}{Yes} & \multicolumn{1}{c}{Yes} & \multicolumn{1}{c}{Yes} & \multicolumn{1}{c}{Yes} & \multicolumn{1}{c}{Yes} & \multicolumn{1}{c}{Yes} & \multicolumn{1}{c}{Yes} & \multicolumn{1}{c}{Yes} & \multicolumn{1}{c}{Yes} \\
Party FE         & \multicolumn{1}{c}{No}  & \multicolumn{1}{c}{No}  & \multicolumn{1}{c}{No}  & \multicolumn{1}{c}{No}  & \multicolumn{1}{c}{No}  & \multicolumn{1}{c}{No}  & \multicolumn{1}{c}{No}  & \multicolumn{1}{c}{No}  & \multicolumn{1}{c}{No}  & \multicolumn{1}{c}{No}  \\
AIC              & 18.00                   & 1134.87                 & 1264.37                 & 1042.07                 & 57.30                   & 56.69                   & 61.28                   & 38.40                   & 39.19                   & 43.55                   \\
BIC              & 52.05                   & 1168.92                 & 1298.42                 & 1098.83                 & 87.57                   & 86.96                   & 102.91                  & 68.67                   & 69.46                   & 85.18                   \\
Log Likelihood   & 0.00                    & -558.43                 & -623.18                 & -506.04                 & -20.65                  & -20.35                  & -19.64                  & -11.20                  & -11.59                  & -10.78                  \\
Deviance         & 180.36                  & 195.93                  & 309.02                  & 91.14                   & 41.30                   & 40.69                   & 39.28                   & 22.40                   & 23.19                   & 21.55                   \\
Num. obs.        & 325                     & 325                     & 325                     & 325                     & 325                     & 325                     & 325                     & 325                     & 325                     & 325                     \\
\bottomrule
\end{tabular}
\begin{tablenotes}[flushleft]
\scriptsize{\item $^{***}p<0.001$; $^{**}p<0.01$; $^{*}p<0.05$. Standard errors in parentheses.
\item Dependent variable: candidate $i$'s list rank (columns 1-4) dual listing status (columns 5-7), and whether the candidate has a tie on the list (columns 8-10).
\item Estimated models: negatige binomial (columns 1-4) and logit (columns 5-10).}
\end{tablenotes}
\end{threeparttable}
}
\caption{Regression Results for Komeito Candidates}
\label{tab:komeito}
\end{center}
\end{table}


\newpage

\subsubsection*{JCP}


\begin{table}[!bth]
\begin{center}
\begin{threeparttable}
\begin{tabular}{l D{.}{.}{4.5} D{.}{.}{4.5} D{.}{.}{4.5} D{.}{.}{4.5} D{.}{.}{4.5}}
\toprule
 & \multicolumn{2}{c}{Dual Listing} & \multicolumn{3}{c}{List Rank} \\
\cmidrule(lr){2-3} \cmidrule(lr){4-6}
 & \multicolumn{1}{c}{H1} & \multicolumn{1}{c}{H2} & \multicolumn{1}{c}{H3} & \multicolumn{1}{c}{H4} & \multicolumn{1}{c}{H5} \\
\midrule
Total Wins      & -0.07                   &                         & -0.24^{***}             &                         &                         \\
                & (0.06)                  &                         & (0.02)                  &                         &                         \\
Incumbency      &                         & -0.32                   &                         & -1.04^{***}             &                         \\
                &                         & (0.25)                  &                         & (0.09)                  &                         \\
Dual Listing    &                         &                         &                         &                         & -0.08                   \\
                &                         &                         &                         &                         & (0.06)                  \\
Female          & -0.09                   & -0.08                   & 0.08                    & 0.07                    & 0.18^{**}               \\
                & (0.22)                  & (0.22)                  & (0.06)                  & (0.06)                  & (0.06)                  \\
Block Magnitude & 0.07^{***}              & 0.07^{***}              & 0.05^{***}              & 0.05^{***}              & 0.05^{***}              \\
                & (0.02)                  & (0.02)                  & (0.00)                  & (0.00)                  & (0.00)                  \\
\midrule
Year FE         & \multicolumn{1}{c}{Yes} & \multicolumn{1}{c}{Yes} & \multicolumn{1}{c}{Yes} & \multicolumn{1}{c}{Yes} & \multicolumn{1}{c}{Yes} \\
Party FE        & \multicolumn{1}{c}{Yes} & \multicolumn{1}{c}{Yes} & \multicolumn{1}{c}{Yes} & \multicolumn{1}{c}{Yes} & \multicolumn{1}{c}{Yes} \\
AIC             & 557.46                  & 557.64                  & 1480.94                 & 1446.81                 & 1623.53                 \\
BIC             & 602.06                  & 602.24                  & 1529.60                 & 1495.46                 & 1672.18                 \\
Log Likelihood  & -267.73                 & -267.82                 & -728.47                 & -711.40                 & -799.76                 \\
Deviance        & 535.46                  & 535.64                  & 223.94                  & 189.81                  & 366.45                  \\
Num. obs.       & 426                     & 426                     & 426                     & 426                     & 426                     \\
\bottomrule
\end{tabular}
\begin{tablenotes}[flushleft]
\scriptsize{\item $^{***}p<0.001$; $^{**}p<0.01$; $^{*}p<0.05$. Standard errors in parentheses.
\item Dependent variable: candidate $i$'s list rank (H1-2) and dual nomination status (H3-5).
\item Estimated models: logit (H1-2) and negative binomial (H3-5).}
\end{tablenotes}
\end{threeparttable}
\caption{Regression Results for JCP Candidates}
\label{tab:regJCP}
\end{center}
\end{table}


\end{document}































